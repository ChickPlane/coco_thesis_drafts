%%%%%%%%%%%%%%%%%%%%%%%%%%%
\chapter {Definitions and Terminology}
\label{PM}
%%%%%%%%%%%%%%%%%%%%%%%%%%%
In this Chapter, we introduce the model, the problem, and some terminology that will be used in the rest of the thesis. We also  describe a general high level technique that is common to most of our solutions.

\section{Model}
\subsection{Network, Agent, Black Virus}

\noindent{\bf Network.} The environment in which mobile agents operate is a network modelled as a simple undirected connected graph with $n = \left |V\right |$ nodes (or sites) and $m = \left |E\right |$ edges (or links). We denote by $E (v)\subseteq E$ the set of edges incident on $v\in V$, by $d (v) = \left|E (v)\right|$ its degree, and by $\Delta(G)$ (or simply $\Delta$) the maximum degree of G. Each node $v$ in the graph has a distinct $id(v)$. The links incident to a node are labelled with distinct port numbers. The labelling mechanism could be totally arbitrary among different nodes; without loss of generality, we assume the link labelling for node $v$ is represented by set $l_v =1,2,3,...,d(v)$. 

\noindent{\bf Agents.} A group of mobile agents is employed to decontaminate the network. An agent is modelled as a computational entity moving from a node to neighbouring node. More than one agents can be at the same node at the same time. Communication among agents occurs when they are at the same node; there are no a priori restrictions on the amount of exchanged information. When the agent arrives at a node, it can leave a message on that node and read a message on that node. Information on the nodes can be set  at the beginning of the exploration (writing the information on a whiteboard), so when an agent reaches a node, it can update the information on the whiteboard or update its own memory by learning the information on the whiteboard.  \color{blue} ? We assume that all the agent' s moves follow the same clock. ? \color{black} 
Also, unless otherwise stated, agents are endowed with 1-hop visibility, which means that, at a node {\em v}, an agent can see the labels of the edges incident to it and the identities of all its 1-hop neighbours. Also, it can   see whether or not there are agents there but cannot communicate with each other. 

\noindent{\bf Black Virus.}
In {\em G} there is a node infected by a black virus (BV) which is a harmful process endowed with reactive capabilities for destruction and spreading. The location of the BV is not known at the beginning. It is not only harmful to the node where it resides but also to any agent arriving at that node. In fact, a BV destroys any agent arriving at the network site where it resides, just like a black hole. Instead of remaining static as the black hole, the BV will spread to all the neighbouring sites leaving the current node clean. The clones can have the same harmful capabilities of the original BVs (fertile) or unable to produce further clones (sterile). A BV will be destroyed if and only if the BV arrives at a node where there is already an agent. Thus, the only way to eliminate the BV is to surround it completely and let an agent attack it. In such a situation, the attacking agent will be destroyed while the clones of the original BV will be permanently eliminated by the agents residing the neighbouring nodes of the original node. We assume that at the same node, multiple BVs (clone or original) are merged. More precisely, at any time, there is at most one BV at each node. 
Another important assumption is that when a BV and an agent arrive at an empty node at the same time, the BV dies and the agent survive remaining unharmed.

Also, in this thesis, we assume that when a BV is triggered,  its clones instantaneously spread to all its neighbour.   That is, when a BV is triggered by an agent at T($t_i$), then at the same time  T($t_i$), all the neighbours of this agents (if any) receive its clones. The reason we make this assumption is that when we try to eliminate the new formed BVs parallelly in the chordal ring and assume that it takes one unit of time for both the agents and the BV to move from one node to another, then we are faced with a tricky situation: if the sites of two (or more) agents are connected, after these two clones are triggered(we send an agent to each of them to permanently destroy them), one of their clones spread to another site and since the agents sent to contaminate them die, these two sites are empty when the second round clones arrive, which make the decontamination invalid.
While with the assumption, the tricky situation can be easily solved: after the two clones are triggered, one of their clones spread to another site and both the original clone and the second round clone are destroyed by the agent.



Summarizing, there are five possible situations when an agent arrive at a node {\em v}:
\begin{itemize}
\item agents arrive at a node which is empty or contains other agents, they can communicate with each other and the node {\em v} is clean.
\item agents arrive at a node which contains a BV, the clones of the BV (BVC) spread to all the neighbours of {\em v} and the agent dies, leaving node {\em v} clean.
\item A BVC arrives at a node which is empty or there is already a BV: the node becomes/stays contaminated; it merges with other BVs.
\item BVCs arrive at a node {\em v} which contains one or more agents, the BVCs are destroyed but the agents are unharmed.
\item A BVC and an agent arrive at an empty node at the same time, the BVC dies while the agent remains unharmed. 
\end{itemize}

\subsection{Problem and Cost Measures}
The {\sc Black Visur Decontamination}  (BVD) problem is to permanently remove the BV  and its clones from the network using a team of mobile agents starting from a given node, called homebase (HB). 
%The solutions where the agents explore the network sequentially have been proposed in some classes of topologies. chordal rings, hypercubes and arbitrary graph. 
%In this thesis, we are interested in parallel strategies for the BVD problem: instead of exploring the network sequentially, as it has been done in the previous literature, we would like to explore  it in parallel; in chordal ring, we also propose a parallel solution to surround the clones of the original BV. 

In this thesis, the cost measures  are the following: 
\begin{itemize}
\item the {\em spread} of the BV: number of nodes that receive a BV or a BVC. This measure also evaluates the loss of   agents (or {\em casualties});
\item  the {\em size} of the team, i.e, the number of agents employed by the solution; 
\item the {\em total working time} (TWT), calculated by multiplying the {\em size} of the team and the {\em time} cost of  the solution. 
%Note that TWT does not contain  any practical meaning but exist only as a measurement. 
We propose TWT to compare more fairly the time of two protocols when the number of agents is different.
\end{itemize}

\subsection{Monotonicity and  Synchrony}
A desirable property of a decontamination protocol is to prevent the nodes that have  been explored or cleaned by mobile agent from being recontaminated,  which would occur if the clones of the BV are able to move to an empty  already  explored node. A BVD protocol with such property is called {\em monotone}. The monotonicity   property is a necessary condition for spread optimality.

Asynchrony refers to the execution timing of agent movement and computations. The timing can be {\em Synchronous} or {\em Asynchronous}. When the timing is synchronous, there is a global clock indicating discrete time units; it takes one unit for each movement (by agent or BV); computing and processing is negligible. When we have asynchronous agents, there is no global clock, and the duration of any activity (e.g., processing, communication, moving) by the agents, the BV, and its clones, is finite but unpredictable. In this thesis, all our protocols work in synchronous setting.

\section{ General strategy}
Following the solution of  sequential case, we decomposed the BVD process into two separate phase: {\em Shadowed Exploration} and {\em Surrounding and Elimination}. The task of the first phase is to locate the BV and the second phase is to decontaminate the BV and its clones.  Additionally to   these two basic phases, we have and  {\em Initialization} phase, which is to properly deploy the agents  before starting the  execution of the protocol. In fact, since the network is explored in parallel,  the initial  arrangement of the agents is crucial to successfully execute the protocol. 

\noindent{\bf Phase1: Shadowed Exploration.}  The agents employed are divided into two groups: shadowing group and exploring group, and the number of agents in  the two group is the same. For convenience, we call the agents in the shadowing group the shadow agents ({\em SA}) and those in the exploring group the exploring agents ({\em EA}) (one {\em EA} is always accompanied by one {\em SA}). As the name indicates, agents in the exploring group explore the network and   agents in shadowing group follow the exploring agents  protecting the nodes that  have been already explored. 
%More precisely, at  time $t_1$, EA moves to node {\em v}; at $t_2$, SA moves to node {\em v} and EA moves to node {\em u} supposing that node {\em v} is clean. 

\noindent{\bf Phase2: Surrounding and Elimination.}
In this phase, since we already know the position of the BV, we employ agents to surround all the neighbours of the BVCs. Once all the agents arrive  at the proper positions (all the neighbours of the BVCs are guarded), we employ another group of agents (the number of them is equal to the number of BVCs) to move to the BVCs, thus permanently  some from the  exploring group because in this way we can save the total number of agents used in the whole process. Note that not all the agents are informed when the BV is detected. More specifically, only the agents that receive the clones know the existence of the BV, while the other agents keep moving in the network. In some simple topologies, such as meshes and tori, the second phase begins when the BV is detected since the number of agents are sufficient to proceed the second phase. In some more complicated topologies, for example, in the chordal ring which we discuss later, we take some other measures to call back enough  agents to finish the second phase.
%\section{Evaluation Criteria}
%For each of our strategy, we compute the number of agents, the number of movements, the time and the casualties.  In addition, we propose another cost measure which combines all the criteria mentioned above (Cost Function Measurement). In this thesis, we propose parallel strategy which aim at reducing the execution time and casualties. We reach this purpose at the cost of employing more agents to explore the graph parallelly. Unlike the sequential strategy which use only one agent, the parallel strategy are more flexible because you can adjust the strategy based on your resource (the number of agents, the time,\ldots). Generally, the more agents we employ to explore the graph, the less time we use in the shadowed exploration phase( the number of casualties may reduce at the same time), so there are some tradeoff we should make when we want to use the parallel BV decontamination strategy in practice. Now we introduce the general idea of the Cost Function Measurement.
%Let's denote by $w$ the number of agents we employ in the strategy, by $x$ the number of movements, by $y$ the time and by $z$ the casualties. Additionally, we introduce three useful function: $Cost_{AgentNumber}(w)$ which given the number of agents employed in the strategy, returns the overall cost of these agents; $Cost_{Movements}(x)$ which given the number of movements needed to complete the strategy, returns the overall cost of the movements; $Cost_{time}(y)$ which given the time required to finish the protocol, returns the overall cost of the time; $Cost_{casualty}(z)$ which given the casualty of the strategy, returns the overall of the casualties. Then the Cost Function $C(w, x, y, z)$ can be presented as $C(w, x, y, z)=C(Cost_{AgentNumber}(w), Cost_{Movements}(x), Cost_{time}(y), Cost_{casualty}(z))$.
%Users can build their own function to fit the specific scenario. For example, if the user care more about the time cost, then they can increase the weight of the time; they can also adjust each subfunction to simulate the practical situation. The simplest model is the liner function. More specifically, the cost functions can be expressed as below:\\
%\begin{equation}
%\left\{
%\begin{aligned}
%&C(w, x, y, z)=Cost_{AgentNumber}(w)+Cost_{Movements}(x)+Cost_{time}(y)+Cost_{casualty}(z)\\
%&Cost_{AgentNumber}(w)=\alpha w+\alpha _1\\
%&Cost_{Movements}(x)=\beta x+\beta _1\\
%&Cost_{time}(y)=\gamma y+\gamma _1\\
%&Cost_{casualty}(z)=\delta z+\delta _1\\
%&w, x, y, z\in \mathbb{Z}\\
%\end{aligned}
%\right.
%\end{equation}
%In this function we can see that the user view the four factor evenly (the coefficients of the four subfunctions are the same which are ``1''), and the cost of each item( agents, movement, time, casualty) increase linearly with their independent variable. We use this liner model to analyze some of our strategies. Note that the function we present here are for the purpose of explaining how to use the function method to analyze the BVD strategies, the function itself can be adjust by the users based on the practical situation.
%We simply compare the overall cost of different strategies for the same problem and assume that the strategy with the minimum cost is the best solution for the users.  
%\section{Conclusion}
%In this Chapter, we presented the model of our problem and   some   terminology. Also we described a general strategy for our problem depending on particular setting: synchronous timing, parallel strategy... In the next chapter, we discuss the parallel strategies in BVD problem in two simple topologies: meshes and torus. 







 



