% A B S T R A C T
% ---------------

\begin{center}\textbf{Abstract}\end{center}



In this thesis, the problem of decontaminating networks from {\em black virus} (BVs) using parallel strategy with a team of system mobile agents (the BVD problem) is studied.
The BV is a harmful process whose initial location is unknown a priori. It destroys any agent arriving at the network site where it resides, and once triggered, it spreads to all the neighboring sites, i.e, its clones, thus increasing its presence in the network. In order to permanently remove any presence of the BV with as less execution time as possible and minimum number of site infections (and thus casualties), we propose parallel strategy to decontaminate the BVs: instead of exploring the network step by step we employ a group of agents who follow the same protocol to explore the network at the same time, thus dramatically reducing the time needed in the exploration phase and minimizing the casualties. Different protocols are proposed in meshes, tori, and chordal rings following the monotonicity principle. Then we analyze the cost of all our solutions and compare to the asynchronous BV decontamination. 
Finally conclusion marks are presented and future researches are proposed.
 

 



\cleardoublepage
%\newpage
