%%%%%%%%%%%%%%%%%%%%%%%%%%%
\chapter {Literature Review}
\label{RW}
%%%%%%%%%%%%%%%%%%%%%%%%%%%
Mobile agents have been widely used in the field of distributed computing due to their features especially the mobility which allow them to migrate between computers at any time during their execution. A group of agents can be used to perform a various tasks, for example, network exploration, maintenance, and etc. However, the introduction of mobile agent tend to cause security problem, thus threatening the network. Various security issues and solution algorithms have been proposed by Flocchini and Santoro in \cite{security}. Generally, the threaten that the mobile agents cause are divided into two categories: in first case,the malicious agents can cause network nodes malfunction or crash by contaminating or infecting them (the harmful agent); in second case, the contaminated or infected hosts can destroy working agent for various malicious purposes (the harmful hosts). These two threaten trigger two problems: Black Hole Search (BHS) and Intruder Capture (IC) which will be introduced in the following sections. Then we review the BVD problem which deal with the decontamination of a harmful presence which cause the network node malfunction but leaves the network node clean when it is triggered and spreads to all its neighbouring nodes, thus increases it presences. In the section introducing BVD problem, we present the the abilities of mobile agents that has been proposed and different decontamination strategies based on different strategies.

\section{Black Hole Search, BHS}
The BHS problem assumes there is a BH or multiple static BHs residing at certain network nodes and will destroy any upcoming agents without leaving any detectable trace.The task is to use a team of agent to locate the black hole(s) and is completed when at least one agent survives and reports the location(s) of the black hole(s). Note that the solution is based on graph exploration and the goal can be reached totally depending on the sacrifice of some agents. In \cite{Das}, Das et al. considered a model for unknown environment with dispersed agents under the weakest possible setting, many exploration models and works were included in this article. The BHS problem has been widely studies in various topologies and settings: the timing is synchronous or asynchronous; the number of black hole(s) is known or unknown. For example: by Chalopin \cite{Das1,Das2} in asynchronous rings and tori, Dobrev et al. \cite{ Dobr} in arbitrary graph, \cite{Dobr1} in anonymous ring and \cite{Dobr2} in common interconnection networks...What is worth pointing out is that the number of BHs remains the same as it of the beginning, thus not causing harm to other sites of the network.


\section{Intruder Capture}
The IC problem assumes that there is an intruder moves with an arbitrary speed from node to node in the network and contaminate the sites it visits, the goal of which is to deploy a group of mobile agents to capture the intruder;the intruder is captured when it comes in contact with an agent. Note that the intruder does not cause any harm to the upcoming agents. It is equivalent to the problem of decontaminating a network contaminated by a virus while avoiding any recontamination. This problem is first introduced in \cite{Flocchini} and has been widely investigated in a variety network topologies: trees \cite{Flocchini, Flocchini1,treeintruder}, hypercubes\cite{Flocchini2}, multi-dimensional grids\cite{ Flocchini2}, chordal rings\cite{Flocchini3} etc. The studies of arbitrary graph has been started in \cite{Nisse,Nisse1}. Note that monotone is a critical principle in the solutions of IC problems.  

\section{Agent Capabilities}
Different capacities granted to the mobile agents have an impart on solving the BHS problem,IC problem and also the BVD problem.now we discuss these capabilities in the following section.

\paragraph{Communication Mechanisms} 
Mobile agents can communicate with each other only when they are in the same node in a network.Some essential communication methods have been studied in literature: whiteboard,tokens and time-out. In \cite{J.C, Dobr, Flocchini4, A.K}, the whiteboard model is used, which is a storage space located at each node and agents arriving there are able to read and write. In the token model, (see \cite{ J.C1, Flocchini4}), tokens are like memos of the agents which can be dropped off and picked by agents at nodes or edges. While the time-out mechanism can only be used in synchronous setting where each agent has a pre-determined amount of time. (see \cite{C.C, C.C1, J.C2}).

\paragraph{Knowledge of the topology} 
Different assumption of mobile agents' knowledge of the topology has an impact on solutions of some of the problems mentioned above, for example, the BHS problem. In \cite{Dobr}, Dobrev et al. present three types of topological knowledge in an asynchronous arbitrary network and show the results of the BHS problem based on different setting of the topology knowledge.

\paragraph{Other capabilities}
In some studies, agents are endowed with the visibility, which mean that they can see whether or not their neighbouring nodes are clean or contaminated (see \cite{M.Huang, M.Huang1}).They observe that the visibility assumption allow them to drastically decrease the time and move complexities in torus, chordal ring and hypercubes when dealing with IC problem. For example, in chordal ring $C_n\{d_1=1,d_2,...,d_k\}$, the number of agents, the time and the moves required in local model are $(2d_k+1)$, $3n-4d_k-1$, $4n-6d_k-1$ respectively, while in visibility model, they are $2d_k$, $\left \lceil \frac{n-2d_{k}}{2(d_{k}-d_{k-1})} \right \rceil$, $n-2d_k$. In torus , the number of agent, the time and the moves required are $2h+1$, $hk-2h$, $2hk-4h-1$ and in visibility model are $2h$, $\left \lceil \frac{k-2}{2} \right \rceil$, $hk-2h$ respectively. They also compare the complexity of both models in hypercubes, a algorithm requiring $\Theta$  ($n \over { \sqrt {log n}} $) number of agents and $O(n \log n)$ moves while the algorithm they propose in the visibility model requires $\frac{n}{2}$ agents and $O(n \log n)$ moves.

In \cite{Cai3}, the concept of {\em k-hop visibility} is presented. The agents have the {\em full topologies} if each of them have a map in their memory  of the entire network including the identities of the node and the labels of the edges. If a agent has {\em k-hop visibility}, then at a node {\em v} a agent can see the k-neighbourhood $N^{(v)}$ of {\em v}, including the node identities and the edge labels. Note that {\em Diam-hop} visibility is equivalent to full topological knowledge. 

Another interesting capability of agents is cloning which is introduced in \cite{M.Huang1}. Cloning is the capacity for an agent to create copies of itself. In this paper, they also discuss how the combination of different capacities reaches different optimal strategy in IC problem in hypercube. For example, the strategy is both time and move-optimal when visibility and cloning are assumed or when cloning, local and synchronicity are assumed. But the time and move-optimal strategy can be obtained at the expense of increasing the number of agents.
The last capability of agents be discussed is immunity which means that a node is immune from recontamination after an agent departs. Two kinds of immunity have been proposed:local and temporal. In local immunity, (see
\cite{Flocchini6,N.Santoro}, the immunity of a node depends on the state of its neighbouring nodes. More specifically, a node remains clean after the departure of an agent until more than half of its neighbours are contaminated. In the temporal immunity, a node is immune for a specific amount of time $(t)$. The node remains clean until time expires and becomes recontaminated if at least one of its neighbours are contaminated. In models without immunity assumption, a node becomes recontaminated if it has at least one contaminated neighbours.

\section{Black Virus Decontamination}
\subsection{Overview}
The BVD problem is first introduced by Cai at al. in cite{Cai}: A black virus is a  extraneous harmful process endowed with capabilities for destruction and spreading. The location of the initial BV(s) is known a priori. Like a BH, a BV destroy any agents arriving at the network where it resides. When that happen, the clones of the original BV spread to all it neighbouring nodes and remain inactive until an agent arrives. The BVD problem is to permanently remove any BVs in the network using a team of mobile agents. They proposed that the only way to decontaminate a BV is to surround all its neighbouring nodes and send an agent to the BV node. In this case, the node where the original BV resides is clean and all its clones are destroyed by the guarding agents in its neighbouring nodes.They have presented different protocols in various topologies: q-grid, q-torus, hypercudes in \cite{Cai} and arbitrary graph \cite{Cai1}. A basic idea of implementing the decontamination has also been propose by them assuming that the timing is asynchronous which divides the whole decontamination process into two part: '{\em shadowed exploration}' and '{\em surround and eliminate}'. In order to minimize the spread of the virus, they use a '{\em safe-exploration}' technique which is executed by at least two agents: the {\em Explorer Agent} and the {\em Leader Explorer Agent} who both residing at a safe node {\em u} at the beginning, for example,the homebase. The {\em Explorer Agent} moves to a node {\em v} to explore it and it needs to return to node {\em u} to report the node {\em v} is safe. The {\em Leader Explorer Agent} determines if the node {\em v} is safe or not by {\em Explorer Agent}'s arriving or a BV's arriving. If node {\em v} is safe, both of them move to node v. For the purpose of insuring monotonicity, at any point in time the already explored nodes must be protected so they are not be recontaminated again. After the BV is detected, the '{\em surround and eliminate}' begins. In this phase, some agents are employed to surround the new-formed BVs (the clones of the original BV) then some agents are sent to the clones to permanently destroy them. This is called the '{\em Four-step Cautious Walk}' and is widely used in BVD problem with synchronous setting. Also, BVD problem in chordal ring has been discussed in
\cite{Alotaibi}.

\subsection{BVD in different topologies}
Protocols regarding to BVD problems in grid are BVD-2G and BVD-qG which deal with BVD problems in 2-dimensional grid (meshes) and q-dimensional grid respectively. BVD-2G performs a BV decontamination of a 2-dimensional grid of size $n$ using $k=7$ agents and 3 casualties, within at most $9n+O(1)$ moves and at most $3n$ time. While protocol BVD-qG performs a decontamination of a q-dimensional grid of size $d_1\times d_2 ...\times d_q$ using $3q+1$agents and at most $q+1$ casualties, within at most $O(qn)$ moves and at most $\Theta$($n$) time. Algorithm to decontaminate the BV in a q-dimensional torus,called BVD-qT uses $4q$ agents with 2 casualties with at most $O(qn)$ moves and $\Theta$($n$). Protocol BVD-qH is to perform a BV decontamination of a q-hypercube using $2q$ agents and $q$ casualties with at most O($n\log n$) moves and $\Theta$($n$). In arbitrary graph (see \cite{Cai1}), two protocols are presented: GREEDY EXPLORATION and THRESHOLD EXPLORATION. In these two protocols, $\Delta +1$ agents are needed and both of the protocols are worst-case optimal with respect to the team size. Though the protocols are described for a synchronous setting, they easily adapt to asynchronous ones with an additional O($n$) moves for the coordinating activities. An advantage of these protocols is that the agents can use only local information to execute the protocol. Another interesting fact based on these two protocols is that both GREEDY ROOTED ORIENTATION and THRESHOLD ROOTED ORIENTATION produce an optimal acyclic orientation rooted in the homebase.

In \cite{Alotaibi}, solution for BVD in chordal ring is discussed. In Alotaibi's thesis, she discuss solutions based on different kinds of chordal ring: double loops, triple loops, consecutive-chords rings and finally general chordal ring. In double loops, she proposed three strategies in elimination phase and the upper bound of moves is $4n-7$ in the whole protocol and a maximum of 12 agents are employed.In triple loops, she discusses two classes of chordal ring: $C_n(1,p,k)$ and $C_n(1,k-1,k)$. In any triple loop $C_n(1,p,k)$, a maximum of $5n-6k+22$ moves and 24 agents are needed for the decontamination while in any triple loop $C_n(1,k-1,k)$, a maximum of $5n-7k+22$ moves and 19 agents are needed. Finally in the consecutive-chords ring, a maximum of $(k+2)n-2k-3$ moves and $4k+1$ agents are needed. She described the decontamination strategies in synchronous setting but only with a cost of $O(n)$ moves can the strategies be used in asynchronous setting.







 









