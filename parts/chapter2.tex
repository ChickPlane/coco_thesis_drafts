%%%%%%%%%%%%%%%%%%%%%%%%%%%
\chapter {Literature Review}
\label{RW}
%%%%%%%%%%%%%%%%%%%%%%%%%%%
Mobile agents have been widely used in  distributed computing due to their features, especially   mobility, which allow them to migrate between computers at any time during their execution. A group of agents can be used to perform a variety of tasks, for example, network exploration, monitoring,  maintenance,   etc. However, the introduction of mobile agent tend to cause security problems, possibly  threatening the network. Various security issues and solution algorithms have been proposed by Flocchini and Santoro in \cite{security}. Generally, the threat  that the mobile agents cause are divided into two categories: in first case, the malicious agents can cause network nodes malfunction or crash by contaminating or infecting them (harmful agents); in second case, the contaminated or infected hosts can destroy working agent for various malicious purposes (harmful hosts). These two threats trigger two problems: Black Hole Search (BHS) and Intruder Capture (IC) which will be introduced in the following Sections. We then  review the BVD problem, which deals with the decontamination of a harmful presence that cause the network node malfunction, leaves the network node clean when it is triggered, and spreads to all its neighbouring nodes increasing its presences. In the section introducing BVD problem, we describe the solutions that have been proposed to decontaminate the networks, and review different decontamination algorithms based on different strategies.

\section{Black Hole Search, BHS}
The BHS problem assumes there is a BH or multiple static BHs residing at certain network nodes. The BHs will destroy any upcoming agents without leaving any detectable trace.The task is to use a team of agents to locate the black hole(s). The task is completed when at least one agent survives and reports the location(s) of the black hole(s). Note that this task is dangerous for the agent;  the solution is based on graph exploration and the goal can be reached totally depending on the sacrifice of some agents and on the observation and deduction by surviving agents {\bf ???}.  In \cite{Das}, Das et al. considered a model for unknown environment with dispersed agents under the weakest possible setting, many exploration models and works were included in this article. {\bf ???} Their research includes different ways to mark the node and to communicate among agents (pebble, marker, and whiteboard), one or multiple agents, asynchronous agents, different topologies (ring, tree, directed or undirected) and different agent memory size limitation and etc.

The BHS problem has been widely studied in various topologies and settings: 
%the timing can be synchronous (there is a global notion of time and an agent movement along a link takes 1 time unit), or asynchronous (no assumption is made on the time taken by each agent to traverse a link and there is no global clock)); the number of black hole(s) can be known or unknown. For example: 
by Chalopin \cite{Das1,Das2} in asynchronous rings and tori, Czyzowicz et al. \cite{czyz} in directed graphs, Dobrev et al. \cite{ dobr} in arbitrary graph, \cite{dobr1} in anonymous ring and \cite{dobr2} in common interconnection networks, .Cooper et. al. \cite{coop} using multiple agents, and Glaus \cite{glau} locating a blacking hole without using the knowledge of incoming link. What is more, Czyzowicz et al. \cite{czyz1} studies the complexity of searching for a black hole, Klasing et. al. \cite{klas} investigated the approximation results and hardness results about the black hole search in arbitrary networks, and \cite{klas1}investigated the approximation bounds for black hole search problems.  

The main difference made in the literature is that whether the system is synchronous or asynchronous. In synchronous model, one agent is sent to explore a node, and other agents can know whether the node is safe or not by waiting until a bounded timer expires. Synchronous model allows detection of unknown number of multiple BH's. In asynchronous model, there is not bounded agent's travelling time on link, so there is no way to distinguish between a slow link and black hole. The ``cautious walk" is the main exploring method to check if the node is safe or not. If one agent is sent to explore to a node, before the node is confirmed safe, no more agents are sent to the node. Asynchronous model only allows detection of known number of BH's with an additional assumption that the number of nodes in the topology is known and if the sum of the explored nodes and the number of the BH's is equal to the number of nodes in the network. In \cite{dobr,dobr1,shi}, they study a variation of model where communication among agents is achieved by placing tokens on the nodes. In \cite{chal2, floc}, they investigate the case of black lines in arbitrary networks for anonymous and non-anonymous nodes. In synchronous setting, Czyzowicz et al.\cite{Czyz2} located a black hole in a synchronous tree network for a given starting node and the minimum number of agents to locate a black hole is two. Cooper et al.\cite{coop1} consider locating  and repairing faults in network. The agent would die after repairing the fault. They present that the number of step sufficient to complete this task is $\Theta(n/k+D)$, where $n$ is the number of nodes in the network and $D$ is the diameter of the network. Finally, some recent studies dealt with scattered agents searching for a black hole in rings and toris \cite{chal,chal1,}. Also, Hohl, Ng, Sander et. al in \cite{ng, hohl1, hohl2, rubi} discussed various methods to protect mobile agent against malicious host. Note that the BH is static which means that it does not propagate in the network and thus not harmful to other sites. The number of BHs does not increase or decrease. 


What is worth pointing out is that the number of BHs remains the same as it of the beginning, thus not causing harm to other sites of the network.
 {\bf A bit more...}

\section{Intruder Capture}
The IC problem assumes that there is an intruder moving with an arbitrary speed from node to node in the network and contaminating the sites it visits. The goal is to deploy a group of mobile agents to capture the intruder; the intruder is captured when it comes in contact with an agent. Note that the intruder does not cause any harm to the upcoming agents. This problem is equivalent to the problem of decontaminating a network contaminated by a virus while avoiding any recontamination which is also called connected graph searching and has been extensively studied (see \cite{fomin}).  The Intruder Capture problem is first introduced in \cite{Flocchini} and has been widely investigated in a variety network topologies: trees \cite{Flocchini, Flocchini1,treeintruder}, hypercubes\cite{Flocchini2}, multi-dimensional grids\cite{Flocchini2}, pyramids\cite{Shar},chordal rings\cite{Flocchini3}, tori\cite{Floc3}, outerplanar graphs\cite{Iman}, etc. The studies of arbitrary graph has been started in \cite{Nisse,Nisse1}. {\bf ?????} Note that monotone is a critical principle in the solutions of IC problems.  
 {\bf A bit more...}

\section{Agent Capabilities}
Different capacities granted to the mobile agents have an impart on solving the BHS problem, IC problem and also the BVD problem.We discuss these capabilities in the following section.

\paragraph{Communication Mechanisms} 
Mobile agents can communicate with each other only when they are in the same node in a network. 
Various  communication methods have been studied in literature: whiteboard, tokens and time-out.  

 In \cite{J.C, Dobr, Flocchini4, A.K}, the whiteboard model is used, which is a storage space located at each node and agents arriving there are able to read and write. In the token model, (see \cite{ J.C1, Flocchini4}), tokens are like pebbles that    the agents can drop   off and pick up  at nodes or edges. Time-out mechanisms achieve implicit communication (or synchronization) among the agents but  can only be used in synchronous settings. (see \cite{C.C, C.C1, J.C2}).

\paragraph{Knowledge of the topology} 
Different assumptions on the agents' knowledge of the topology have an impact on solutions of some of the problems mentioned above, for example, the BHS problem. In \cite{Dobr}, Dobrev et al. present three types of topological knowledge in an asynchronous arbitrary network and show the results of the BHS problem based on different setting of the topology knowledge.

\paragraph{Other capabilities}
In some studies, agents are endowed with   visibility, which means that they can see whether or not their neighbouring nodes are clean or contaminated (see \cite{M.Huang, M.Huang1}).It is observed that the visibility assumption allows   to drastically decrease the time and move complexities in tori, chordal rings and hypercubes when dealing with IC problem. For example, in chordal ring $C_n\{d_1=1,d_2,...,d_k\}$, the number of agents, the time and the moves required in local model are $(2d_k+1)$, $3n-4d_k-1$, $4n-6d_k-1$ respectively, while in the visibility model, they are $2d_k$, $\left \lceil \frac{n-2d_{k}}{2(d_{k}-d_{k-1})} \right \rceil$, $n-2d_k$. In tori, the number of agents,   time and   moves required are, respectively,  $2h+1$, $hk-2h$, $2hk-4h-1$, while  in the  visibility model they are $2h$, $\left \lceil \frac{k-2}{2} \right \rceil$, $hk-2h$, respectively. The authors also compare the complexity of both models in hypercubes  and propose an algorithm that requires $\Theta$  ($n \over { \sqrt {log n}} $) agents and $O(n \log n)$ moves, and another algorithm   in the visibility model that requires $\frac{n}{2}$ agents and $O(n \log n)$ moves.

In \cite{Cai3}, the concept of {\em k-hop visibility} is presented. The agents have   {\em full topology} knowledge if each of them have a map in their memory  of the entire network including the identities of the node and the labels of the edges. An agent has {\em k-hop visibility}, when at a node {\em v} the agent can see the k-neighbourhood $N^{(v)}$ of {\em v}, including the node identities and the edge labels. Note that {\em Diam-hop} visibility is equivalent to full topological knowledge. 

Another interesting capability of agents is cloning, which is introduced in \cite{M.Huang1}. Cloning is the capacity for an agent to create copies of itself. In \cite{M.Huang1} it is also discussed how the combination of different capacities allows different optimal strategies for the IC problem in the hypercube. For example, a  time and move optimal algorithm is devised  when visibility and cloning are assumed or when cloning,  and synchronicity are assumed and the model is  local. However, the time and move-optimal strategy is obtained at the expense of increasing the number of agents.
The last capability of agents that has been discussed is immunity, where a node is immune from recontamination after an agent departs under some conditions. Two kinds of immunity have been proposed: local and temporal. In local immunity, (see
\cite{Flocchini6,N.Santoro}, the immunity of a node depends on the state of its neighbouring nodes. More specifically, a node remains clean after the departure of an agent until more than half of its neighbours are contaminated. In the temporal immunity, a node is immune for a specific amount of time. The node remains clean until this time expires and it becomes recontaminated if at least one of its neighbours are contaminated. In models without immunity assumption, a node becomes recontaminated if it has at least one contaminated neighbours.

\section{Black Virus Decontamination}
\subsection{Overview}
The BVD problem is first introduced by Cai at al. in cite{Cai}: A black virus is a  extraneous harmful process endowed with capabilities for destruction and spreading. The location of the initial BV(s) is known a priori. Like a BH, a BV destroys any agent arriving at the network where it resides. When that happen, the clones of the original BV spread to all it neighbouring nodes and remain inactive until an agent arrives. The BVD problem is to permanently remove any BVs in the network using a team of mobile agents. In \cite{Cai}  the strategy to decontaminate a BV is to surround all its neighbouring nodes and send an agent to each of them. In this case, the node where the original BV resides is clean and all its clones are destroyed by the guarding agents in its neighbouring nodes.The authors have presented different protocols in various topologies: q-grid, q-torus, hypercudes  (\cite{Cai}), and arbitrary graph \cite{Cai1}. A basic idea of implementing the decontamination has also been proposed  by   assuming that the timing is asynchronous which divides the whole decontamination process into two part: ``shadowed exploration''and ``surround and eliminate''. In order to minimize the spread of the virus, they use a ``safe-exploration'' technique which is executed by at least two agents: the ``Explorer Agent'' and the ``Leader Explorer Agent''. Both agents  initially resides at a safe node {\em u}called, the {\em homebase}. The Explorer Agent moves to a node {\em v} to explore it and it needs to return to node {\em u} to report the node {\em v} is safe. The ``Leader Explorer Agent" determines if the node {\em v} is safe or not by the arrival of the ``Explorer Agent''. If node {\em v} is safe, both agents move to node $v$. For the purpose of insuring monotonicity, at any point in time the already explored nodes must be protected so that they are not be recontaminated again. After the BV is detected, the ``surround and eliminate'' begins. In this phase, some agents are employed to surround the new-formed BVs (the clones of the original BV) then some agents are sent to the clones to permanently destroy them. This is called the ``Four-step Cautious Walk" and is widely used in BVD problem with synchronous setting. Also, BVD problem in chordal ring has been discussed in
\cite{Alotaibi}.

\subsection{BVD in different topologies}
Protocols to solve the  BVD problems in grid are BVD-2G and BVD-qG which deal with BVD problems in 2-dimensional grid (meshes) and q-dimensional grid respectively. BVD-2G performs a BV decontamination of a 2-dimensional grid of size $n$ using $k=7$ agents and 3 casualties (i.e., losses of agents), within at most $9n+O(1)$ moves and at most $3n$ time. While protocol BVD-qG performs a decontamination of a q-dimensional grid of size $d_1\times d_2 ...\times d_q$ using $3q+1$agents and at most $q+1$ casualties, within at most $O(qn)$ moves and at most $\Theta$($n$) time. Algorithm to decontaminate the BV in a q-dimensional torus,called BVD-qT uses $4q$ agents with 2 casualties with at most $O(qn)$ moves and $\Theta$($n$). Protocol BVD-qH is to perform a BV decontamination of a q-hypercube using $2q$ agents and $q$ casualties with at most O($n\log n$) moves and $\Theta$($n$). In arbitrary graphs (see \cite{Cai1}), two protocols are presented: GREEDY EXPLORATION and THRESHOLD EXPLORATION. In these two protocols, $\Delta +1$ agents are needed and both of the protocols are worst-case optimal with respect to the team size, where $\Delta$ represents the maximum degree in G. Though the protocols are described for a synchronous setting, they easily adapt to asynchronous ones with an additional O($n$) moves for the coordinating activities. An advantage of these protocols is that the agents can use only local information to execute the protocol. Another interesting fact based on these two protocols is that both GREEDY ROOTED ORIENTATION and THRESHOLD ROOTED ORIENTATION produce an optimal acyclic orientation rooted in the homebase.

In \cite{Alotaibi}, solutions for BVD in chordal ring are discussed. The solutions are designed for  different kinds of chordal rings: double loops, triple loops, consecutive-chords rings and finally general chordal ring. In double loops,     three strategies  are proposed  
%for the elimination phase and 
with an  upper bound of moves is $4n-7$  and a maximum of 12 agents   employed.
In triple loops,  wo classes of chordal rings are discussed: $C_n(1,p,k)$  and $C_n(1,k-1,k)$. In any triple loop $C_n(1,p,k)$, a maximum of $5n-6k+22$ moves and 24 agents are needed for the decontamination while in any triple loop $C_n(1,k-1,k)$, a maximum of $5n-7k+22$ moves and 19 agents are needed. Finally in the consecutive-chords ring, a maximum of $(k+2)n-2k-3$ moves and $4k+1$ agents are needed. Decontamination strategies  are described  in synchronous settings, but   with a cost of $O(n)$ moves   the strategies can be used in asynchronous settings as well.







 









