%%%%%%%%%%%%%%%%%%%%%%%%%%%
\chapter {Literature Review}
\label{RW}
%%%%%%%%%%%%%%%%%%%%%%%%%%%
Mobile agents have been widely used in the field of distributed computing due to their features especially the mobility which allow them to migrate between computers at any time during their execution. A group of agents can be used to perform a various tasks, for example, network exploration, maintenance, and etc. However, the introduction of mobile agent tend to cause security problem, thus threatening the network. Various security issues and solution algorithms have been proposed by Flocchini and Santoro in \cite{security}. Generally, the threaten that the mobile agents cause are divided into two categories: in first case,the malicious agents can cause network nodes malfunction or crash by contaminating or infecting them (the harmful agent); in second case, the contaminated or infected hosts can destroy working agent for various malicious purposes (the harmful hosts). These two threaten trigger two problems: Black Hole Search (BHS) and Intruder Capture (IC) which will be introduced in the following sections. Then we review the BVD problem which deal with the decontamination of a harmful presence which cause the network node malfunction but leaves the network node clean when it is triggered and spreads to all its neighbouring nodes, thus increases it presences. In the section introducing BVD problem, we present the the abilities of mobile agents that has been proposed and different decontamination strategies based on different strategies.

\section{Black Hole Search, BHS}
The BHS problem assumes there is a BH or multiple static BHs residing at certain network nodes and will destroy any upcoming agents without leaving any detectable trace.The task is to use a team of agent to locate the black hole(s) and is completed when at least one agent survives and reports the location(s) of the black hole(s). Note that the solution is based on graph exploration and the goal can be reached totally depending on the sacrifice of some agents. In \cite{Das}, Das et al. considered a model for unknown environment with dispersed agents under the weakest possible setting, many exploration models and works were included in this article. The BHS problem has been widely studies in various topologies and settings: the timing is synchronous or asynchronous; the number of black hole(s) is known or unknown. For example: by Chalopin \cite{Das1,Das2} in asynchronous rings and tori, Dobrev et al. \cite{ Dobr} in arbitrary graph, \cite{Dobr1} in anonymous ring and \cite{Dobr2} in common interconnection networks...What is worth pointing out is that the number of BHs remains the same as it of the beginning, thus not causing harm to other sites of the network.


\section{Intruder Capture}
The IC problem assumes that there is an intruder moves with an arbitrary speed from node to node in the network and contaminate the sites it visits, the goal of which is to deploy a group of mobile agents to capture the intruder;the intruder is captured when it comes in contact with an agent. Note that the intruder does not cause any harm to the upcoming agents. It is equivalent to the problem of decontaminating a network contaminated by a virus while avoiding any recontamination. This problem is first introduced in \cite{Flocchini} and has been widely investigated in a variety network topologies: trees \cite{Flocchini, Flocchini1,treeintruder}, hypercubes\cite{Flocchini2}, multi-dimensional grids\cite{ Flocchini2}, chordal rings\cite{Flocchini3} etc. The studies of arbitrary graph has been started in \cite{Nisse,Nisse1}. Note that monotone is a critical principle in the solutions of IC problems.  

\section{Agent Capabilities}
Different capacities granted to the mobile agents have an impart on solving the BHS problem,IC problem and also the BVD problem.now we discuss these capabilities in the following section.

\paragraph{Communication Mechanisms} 
Mobile agents can communicate with each other only when they are in the same node in a network.Some essential communication methods have been studied in literature: whiteboard,tokens and time-out. In \cite{J.C, Dobr, Flocchini4, A.K}, the whiteboard model is used, which is a storage space located at each node and agents arriving there are able to read and write. In the token model, (see \cite{ J.C1, Flocchini4}), tokens are like memos of the agents which can be dropped off and picked by agents at nodes or edges. While the time-out mechanism can only be used in synchronous setting where each agent has a pre-determined amount of time. (see \cite{C.C, C.C1, J.C2}).

\paragraph{Knowledge of the topology} 
Different assumption of mobile agents' knowledge of the topology has an impact on solutions of some of the problems mentioned above, for example, the BHS problem. In \cite{Dobr}, Dobrev et al. present three types of topological knowledge in an asynchronous arbitrary network and show the results of the BHS problem based on different setting of the topology knowledge.

\paragraph{Other capabilities}
In some studies, agents are endowed with the visibility, which mean that they can see whether or not their neighbouring nodes are clean or contaminated (see \cite{M.Huang, M.Huang1}).They observe that the visibility assumption allow them to drastically decrease the time and move complexities in torus, chordal ring and hypercubes when dealing with IC problem. For example, in chordal ring $C_n\{d_1=1,d_2,...,d_k\}$, the number of agents, the time and the moves required in local model are $(2d_k+1)$, $3n-4d_k-1$, $4n-6d_k-1$ respectively, while in visibility model, they are $2d_k$, $\left \lceil \frac{n-2d_{k}}{2(d_{k}-d_{k-1})} \right \rceil$, $n-2d_k$. In torus , the number of agent, the time and the moves required are $2h+1$, $hk-2h$, $2hk-4h-1$ and in visibility model are $2h$, $\left \lceil \frac{k-2}{2} \right \rceil$, $hk-2h$ respectively. They also compare the complexity of both models in hypercubes, a algorithm requiring $\Theta$  ($n \over { \sqrt {log n}} $) number of agents and $O(n \log n)$ moves while the algorithm they propose in the visibility model requires $\frac{n}{2}$ agents and $O(n \log n)$ moves.
Another interesting capability of agents is cloning which is introduced in \cite{M.Huang1}. Cloning is the capacity for an agent to create copies of itself. In this paper, they also discuss how the combination of different capacities reaches different optimal strategy in IC problem in hypercube. For example, the strategy is both time and move-optimal when visibility and cloning are assumed or when cloning, local and synchronicity are assumed. But the time and move-optimal strategy can be obtained at the expense of increasing the number of agents.
The last capability of agents be discussed is immunity which means that a node is immune from recontamination after an agent departs. Two kinds of immunity have been proposed:local and temporal. In local immunity, (see
\cite{Flocchini6,N.Santoro}, the immunity of a node depends on the state of its neighbouring nodes. More specifically, a node remains clean after the departure of an agent until more than half of its neighbours are contaminated. In the temporal immunity, a node is immune for a specific amount of time $(t)$. The node remains clean until time expires and becomes recontaminated if at least one of its neighbours are contaminated. In models without immunity assumption, a node becomes recontaminated if it has at least one contaminated neighbours.

\section{Synchrony}
Asynchrony refers to the execution timing of agent movement and computations. The timing can be {\em Synchronous} or{\em Asynchronous}. When the timing is synchronous, there is a global clock indicating discrete time unit; it takes one unit for each movement (by agent or BV); computing and processing is negligible.
When we have asynchronous agents, there is no global clock, and the duration of any activity (e.g., processing, communication, moving) by the agents, the BV, and its clones is finite but unpredictable.

\section{Black Virus Decontamination}
\subsection{Overview}
The BVD problem is first introduced by Cai at al. in cite{Cai}: A black virus is a  extraneous harmful process endowed with capabilities for destruction and spreading. The location of the initial BV(s) is known a priori. Like a BH, a BV destroy any agents arriving at the network where it resides. When that happen, the clones of the original BV spread to all it neighbouring nodes and remain inactive until an agent arrives. The BVD problem is to permanently remove any BVs in the network using a team of mobile agents. They proposed that the only way to decontaminate a BV is to surround all its neighbouring nodes and send an agent to the BV node. In this case, the node where the original BV resides is clean and all its clones are destroyed by the guarding agents in its neighbouring nodes.They have presented different protocols in various topologies: q-grid, q-torus, hypercudes in \cite{Cai} and arbitrary graph \cite{Cai1}. A basic idea of implementing the decontamination has also been propose by them assuming that the timing is asynchronous which divides the whole decontamination process into two part: '{\em shadowed exploration}' and '{\em surround and eliminate}'. In order to minimize the spread of the virus, they use a '{\em safe-exploration}' technique which is executed by at least two agents: the {\em Explorer Agent} and the {\em Leader Explorer Agent} who both residing at a safe node {\em u} at the beginning, for example,the homebase. The {\em Explorer Agent} moves to a node {\em v} to explore it and it needs to return to node {\em u} to report the node {\em v} is safe. The {\em Leader Explorer Agent} determines if the node {\em v} is safe or not by {\em Explorer Agent}'s arriving or a BV's arriving. If node {\em v} is safe, both of them move to node v. For the purpose of insuring monotonicity, at any point in time the already explored nodes must be protected so they are not be recontaminated again. After the BV is detected, the '{\em surround and eliminate}' begins. In this phase, some agents are employed to surround the new-formed BVs (the clones of the original BV) then some agents are sent to the clones to permanently destroy them. This is called the '{\em Four-step Cautious Walk}' and is widely used in BVD problem with synchronous setting. Also, BVD problem in chordal ring has been discussed in
\cite{Alotaibi}.

\subsection{BVD in different topologies}
Protocols regarding to BVD problems in grid are BVD-2G and BVD-qG which deal with BVD problems in 2-dimensional grid (meshes) and q-dimensional grid respectively. BVD-2G performs a BV decontamination of a 2-dimensional grid of size $n$ using $k=7$ agents and 3 casualties, within at most $9n+O(1)$ moves and at most $3n$ time. While protocol BVD-qG performs a decontamination of a q-dimensional grid of size $d_1\times d_2 ...\times d_q$ using $3q+1$agents and at most $q+1$ casualties, within at most $O(qn)$ moves and at most $\Theta$($n$) time. Algorithm to decontaminate the BV in a q-dimensional torus,called BVD-qT uses $4q$ agents with 2 casualties with at most $O(qn)$ moves and $\Theta$($n$). Protocol BVD-qH is to perform a BV decontamination of a q-hypercube using $2q$ agents and $q$ casualties with at most O($n\log n$) moves and $\Theta$($n$). In arbitrary graph (see \cite{Cai1}), two protocols are presented: GREEDY EXPLORATION and THRESHOLD EXPLORATION. In these two protocols, $\Delta +1$ agents are needed and both of the protocols are worst-case optimal with respect to the team size. Though the protocols are described for a synchronous setting, they easily adapt to asynchronous ones with an additional O($n$) moves for the coordinating activities. An advantage of these protocols is that the agents can use only local information to execute the protocol. Another interesting fact based on these two protocols is that both GREEDY ROOTED ORIENTATION and THRESHOLD ROOTED ORIENTATION produce an optimal acyclic orientation rooted in the homebase.

In \cite{Alotaibi}, solution for BVD in chordal ring is discussed. In Alotaibi's thesis, she discuss solutions based on different kinds of chordal ring: double loops, triple loops, consecutive-chords rings and finally general chordal ring. In double loops, she proposed three strategies in elimination phase and the upper bound of moves is $4n-7$ in the whole protocol and a maximum of 12 agents are employed.In triple loops, she discusses two classes of chordal ring: $C_n(1,p,k)$ and $C_n(1,k-1,k)$. In any triple loop $C_n(1,p,k)$, a maximum of $5n-6k+22$ moves and 24 agents are needed for the decontamination while in any triple loop $C_n(1,k-1,k)$, a maximum of $5n-7k+22$ moves and 19 agents are needed. Finally in the consecutive-chords ring, a maximum of $(k+2)n-2k-3$ moves and $4k+1$ agents are needed. She described the decontamination strategies in synchronous setting but only with a cost of $O(n)$ moves can the strategies be used in asynchronous setting.

Chapter 2 end 


%\section{Decontamination}

%Because of network connectivity, a moving object can contaminate an entire network, resulting in dysfunctional performance. This scenario is considered to be one of the most critical threads in network security (ie.,  viruses and how they attack devices on the Internet). As a result, there is a need for a cleaning and protecting process ({\em Decontamination}). This problem has been extensively studied in graph theory under several names: graph search, intruder capture and decontamination. This problem was first introduced in 1967 by \cite{bre4}. The decontamination process can be conducted internally or externally. Internal decontamination allows a node to perform the task without external help. In other words, the nodes are able to decontaminate themselves locally. For example, running an Anti-virus program on an infected device could rid that device from viruses it got from the Internet. On the other hand, external decontamination requires external intervention, such as mobile agents. An agent decontaminates infected nodes by passing through them, yet the nodes are vulnerable to recontamination once the agents depart. 


The Mobile Agents Model is a form of external decontamination in which moving agents are used to disinfect a contaminated network.
The mobile agents must decontaminate the network in a way that prevents recontamination. In this model, nodes have three states: contaminated, decontaminated or guarded. A node is decontaminated once an agent passes through it, guarded once an agent resides in it and otherwise contaminated. Initially, all nodes are contaminated with the exception of the homebase which is guarded. This model has many variations and parameters that have a significant impact on the overall complexity. As a result, there are many protocols and strategies that handle this problem in various topologies. These protocols were designed to be {\em monotone}, that is, once a node is decontaminated, it can not be re-infected.

Usually, the complexity of such protocols is measured by the number of agents used in the decontamination process, the number of moves the agents make and the amount of time needed to decontaminate an entire network.

The following variations have a significant impact on protocol performance and complexity: synchronicity, agent capabilities and topological characteristics.
\subsection{Synchronicity}
For synchronous execution many papers address the problem of decontamination where agents synchronize their movements in such a way that the explored nodes never get re-contaminated  (see \cite{lucetal22,floc21,baretal24}). In the case of asynchronous execution, there is a need for a coordinator who organizes other agents, ensuring that all agents are in the correct positions before starting a new step (see \cite{floetal13,lucetal22,floetal20,floetal19,floetal17}).


\subsection{Agent Capabilities}
In this problem, agents have been given certain capabilities including cloning, visibility and immunity.

\paragraph{ Visibility.}  Some studies have granted agents the visibility, which is the ability to see whether the neighbouring nodes are clean or contaminated (see \cite{floetal17,floetal20}). Some studies have only employed the local model in which agents only know the information (state, ports) for the node in which they reside (\cite{floetal19,floetal17,floetal20}). Visibility impacts complexity since the agents are able to move in an autonomous fashion, without the need for coordination. In \cite{floetal17}, Flocchini et al. demonstrate that to decontaminate a chordal ring $C_n\{d_1=1,d_2,...,d_k\}$ in the local model, the number of agents required is $(2d_k+1)$, while in the visibility model, the number of agents required is $(2  d_k)$.

 In \cite{floetal20}, Flocchini et al. compare the complexity of  decontaminating a hypercube in both models. In the local model, they propose an algorithm that requires  $\Theta$  ($n \over { \sqrt {log n}} $) number of agents and  $O(n \log n)$  moves. In the  visibility model, they propose an algorithm that requires ($n\over2$) agents and $O(n \log n)$ moves.



\paragraph{ Cloning.}
Cloning refers to an agent's ability to make copies of itself. This feature would reduce the number of moves but would not produce an optimal number of agents. In \cite{floetal20}, Flocchini et al. study the impact of the cloning feature on the decontamination of a hypercube. They demonstrate that using cloning, the number of moves is drastically reduced: ($n-1$) in both the visibility and local models.


\paragraph{ Immunity.}
Immunity means that a node is immune from recontamination after an agent departs. In the literature, there are two types of immunity: local and temporal. In local immunity, the immunity of a node is determined by the state of its neighbours. For instance, a node stays clean after the departure of an agent unless the majority of its neighbours are contaminated. If the majority of its neighbours are contaminated, that node gets re-contaminated. In other words, the immunity level in this model is about half of the node’s degree. If the contamination reaches more than half, the node gets re-contaminated (see \cite{lucetal22,floc21}). In temporal immunity, a node is immune for a specified amount of time $(t)$. After the departure of an agent, the node is clean for $t$ time regardless of the state of its neighbours. After the time elapses, the node is vulnerable to recontamination if it has at least one contaminated neighbour (see \cite{floetal13}). The two types of immunity provide a network with some resistance to recontamination.  In the absense of these two models, the immunity of the nodes is nil, meaning that the node can be recontaminated if any of its neighbours are contaminated.



\subsection{Topologies}

The mobile agent decontamination model has been studied in various topologies. In this section we list the results of decontamination protocols in some common topologies.

 
\noindent{\bf Tree.}  
Many researchers have studied the decontamination process in trees. The basic strategy for decontaminating a tree is to divide the tree into sub-trees starting from a single homebase (the root), see \cite{floetal13,lucetal22}. 

The best case scenario in terms of the number of agents required occurs when the homebase is the first or last node and we have a line that requires one agent for decontamination. If the homebase is any other node, we require two agents. The worst possible case is the decontamination of a complete binary tree.


In \cite{lucetal22}, Luccio et al. consider an asynchronous environment, continuous search, visibility mode, local immunity, single homebase, and a monotone strategy. In order to enforce local immunity there are three basic rules controlling agent departure from a node. If the number of clean neighbours is lower than half of the total number of neighbours (i.e. the majority), the agent does not depart. If the number of clean neighbours is equal to half of the total number of neighbours, the agent only moves to a contaminated neighbour. If the number of clean neighbours is greater than half of the total number of neighbours, the agent moves to any neighbour.
To identify the minimum number of cleaning agents, this paper divides trees into two classes: trees with degree $d \leq 3$ (e.g. binary tree, line) and general arbitrary trees.
\begin{itemize}

\item Trees with degree $d \leq 3$ can be decontaminated using a maximum of two agents,
%Obviously, we need only one agent to decontaminate a line; also we need only one agent to decontaminate any n-node binary tree, 
while the number of moves is $(2(n-1) - diam)$, where $diam$ represents the diameter. 
\item For general arbitrary trees with degree $d >3$, this paper introduces the state of stability. A node x is stable if it satisfies the following two conditions: x is immune to recontamination whether it is guarded or not, and the majority of its children are stable. 
The authors use these conditions to determine the minimum number of cleaners needed to keep a node stable. As a result, we can determine the minimum number of agents required to decontaminate the entire tree. Any lower bound on the minimum number of agents needed to stabilize x provides a lower bound on the minimum number of agents needed to decontaminate a tree rooted in the homebase. In this protocol, the agent stabilizes the root and then recursively moves to unstable nodes and stabilizes them until all nodes are stable. The number of moves in this protocol is not proven. This study demonstrates how local immunity improves complexity (minimum number of cleaners), especially in a binary tree.
 \end{itemize}
In \cite{floetal13}, the authors consider a synchronous environment, continuous search, temporal immunity (i.e. $t>0$), single homebase and a monotone strategy. In this case, any node is immune from recontamination when it is guarded and during the immunity time $t$ after the agent leaves. Therefore, agents can move back and forth during period t. In other words, an agent can leave a node and traverse a path of length up to t/2 and go back to that node without worrying about recontamination. Using this property, the Depth-First traversal strategy, and a decontaminating leader (i.e., an agent), this paper demonstrates that a tree requires at least $2h\over(t+1)$ agents, where $h$ is the height of the tree, and $t$ is the immunity time. We can see that the temporal immunity has a significant impact on the process of decontamination. 


\noindent{\bf  Hypercube.}
The hypercube is thoroughly studied in \cite{floetal20}. This paper discusses several strategies for decontaminating a hypercube. The authors consider the local and visibility models, with and without cloning as we mentioned in the previous section.

%For the first strategy, the authors assume local model, asynchronous environment, single homebase, continuous search, and monotone strategy. This strategy relies on the existence of a coordinator (agent) that directs how the other agents move since the agents don’t have the capability to see the states of neighbours. This decontaminating strategy is performed on the broadcast tree of the hypercube level by level.
%It is proven that this strategy requires number of agents= Θ(n/(log n)1/2) ,and number of moves= O(n log n), where n is the number of nodes.
%The second strategy is the same as the first one except that this strategy considers the visibility model. This strategy uses (n/2) agents and O(n log n) moves to decontaminate the network. We can notice the drop in the number of agents in the visibility model. Lastly, the third strategy considers the visibility model and cloning. However, in this case no coordinator is needed. In this protocol, Initially, one agent starts from the homebase, then it clones agents as many as the number of dimensions -1, and send one agent through each link. After an agent cleans a node, it clones enough agents to clean all neighbors that are not clean or guarded. As a result the number of moves performed by agents is decreased to (n-1), where the minimum number of agents required for monotone decontamination is n/2. The following table shows a comparison between the previous three strategies in terms of complexity. Hence the table shows the time complexity.






\noindent{\bf Mesh.}
As seen in \cite{floetal19}, to decontaminate a mesh (grid) of $M\times N$ nodes, there two possible strategies. The authors assume an asynchronous environment, continuous search, single homebase, and monotone protocol. In the first strategy the authors use the local model in which the agents cannot see neighbouring states but have the ability to exchange information using whiteboard. In this strategy there is a need for a synchronizer that coordinates the protocol execution. Beginning from the homebase, $M$ agents (except one agent that remains at the homebase) move south in order to guard each node in the first column. The synchronizer then coordinates the moves of all agents in such a way that all the agents are moving east at the same time.  This process goes on column by column until the whole graph is decontaminated. This protocol requires $M+1$ agents and ${M^2+4MN-5M-2}\over 2$ moves. In the second strategy, the authors use the visibility model, eliminating the need for a synchronizer. In this case, agents depend on what has been written in the whiteboard to decide the next step. This strategy requires $M$ agents (no synchronizer here) and ${M^2+2MN-3M}\over 2$ moves.

In the previous two protocols, the authors assume a mesh M×N, where $M\leq N$, so the protocols carry on column by column. In the case of  $N < M$, it is better to proceed row by row to obtain a lower number of agents.

\noindent{\bf Tori.}
A torus $h\times k$ ($h<=k$) is a mesh where the nodes in the last row are connected to nodes in the first row, and nodes in the first column are connected to nodes in the last column. In order to decontaminate a torus we can use the strategies used to decontaminate a mesh with some slight modifications.  In \cite{floetal17}, Flocchini et al. use strategies similar to those used for the mesh. Instead of deploying agents to cover one column, agents are deployed to cover two successive columns. One column of agents then stays to guard the nodes from recontamination while the other column of agents moves through the torus. In the local model there is a need for a synchronizer. The complexity of this strategy is $2h+1$ agents and ($2hk-4h-1$) moves. In the visibility model, there is no need for a synchronizer, and the complexity is $2h$ agents and ($hk-2h$) moves. In \cite{lucetal22}, the authors demonstrate that in order to decontaminate a  k-dimensional tori ($k>2$) in the presence of local immunity, we need at least $2k$ synchronous agents and ($n+2k-2k-2$) moves.

\noindent{\bf Rings.}
When we have a ring  it is easy to find the minimum number of agents required for monotone decontamination. 

In the local model, two agents are sufficient to perform the decontamination if they start from a single homebase and move in opposite directions until they are reunited. The number of moves would therefore be n, where n represents the number of nodes, if we assume that an agent needs one time unit to traverse a link. In the visibility model, the two agents move in opposite directions until they reach two consecutive nodes. The number of moves is therefore $n-1$ (see  \cite{lucetal22}).


\noindent{\bf Chordal Rings.}
In \cite{floetal17}, Flocchini et al. provide the results of decontaminating a chordal ring $C_n\{d_1=1,d_2,...,d_k\}$ using the local and visibility models. In the local model, the complexity required to decontaminate the chordal ring using  asynchronous execution is $ (2d_k+1)$ agents and $4n-6d_k-1$ number of moves. In the visibility model, the number of agents is $(2  d_k)$  and the number of moves is ($n-2d_k$). 





%\section{Fault Tolerance in Chordal Rings}

\section{Black Virus Decontamination}
This term was initially introduced by Cai el al. in \cite{caietal18}. The {\it Black Virus} ($BV$) is a hostile node that resides in an unknown location, causing the destruction of any visiting mobile agent. The black virus also causes more damage by moving to neighbouring nodes. Therefore, we need a strategy to locate the hostile node and to disinfect the entire topology from its spread.
In \cite{caietal18}, the authors present the {\it Black Virus} (BV) threat by combining the statical feature of the {\it Black Hole} and the mobility feature of the {\it Intruder Capture}. They also propose a solution to the $BVD$ problem using a team of mobile agents. 

The authors assume that there is initially only one $BV$ in the network. Like $BH$, the location of $BV$ is unknown a priori and the $BV$  stays inactive (unharmful) unless it is triggered. The $BV$ is activated when a mobile agent arrives at its location. The mobile agent is then destroyed and the $BV$ clones itself into as many $BV$s as its number of degrees. The copies retain the same features as the original and each copy moves to a neighbouring node and remains inactive until triggered. According to \cite{caietal18}, the \bv is only deactivated when it moves to a node that is occupied by an agent. To the best of our knowledge, this problem has only been studied by Cai et al. in \cite{caietal18}.

In \cite{caietal18}, the proposed protocol consists of two phases: '{\em shadowed exploration}' and '{\em surround and eliminate}'. As the names suggest, the first phase involves exploring  the network, locating the \bv and triggering it. The second phase involves surrounding the newly created $BV$s and then triggering them to remove them from the network. This protocol is monotone, meaning that once a node is explored, it is immune from recontamination. The measures of efficiency considered include:  the spread ($spread$), the number of agents ($size$) and the number of moves.
The $BVD$ problem is solved when the network is completely decontaminated and at least one agent survives.
 The authors conduct their research using common topologies such as rings, multi-dimensional grids, multi-dimensional tori and hypercubes.

\noindent{\bf Ring(R).} The authors demonstrate that, regardless of the number of nodes ($n$), the monotone protocol requires ($size(R)= 4$) and ($spread(R)= 2$). It is possible that these results are not quite accurate. If the  \bv resides in node $n-1$, we either have a non-monotone protocol in which an explored node gets contaminated or  less complexity where ($size\leq 4$) and ($spread \leq 2$).

\noindent{\bf Grid(G).}The paper considers a 2-dimensional grid and a q-dimensional grid. In a 2-dimensional grid of size $d_1\times d_2$, they prove that regardless of $n$ their proposed optimal algorithm would cost ($spread(G)=3$), ($size(G)=7$), and at most ($5n+O(1)$) number of moves . While in a q-dimensional grid, the complexity would be ($spread(G)=q+1$), ($size(G)=3q+1$), and at most ($O(qn)$)  number of moves. 

\noindent{\bf Torus(T).} According to the protocol, decontaminating a q-dimensional torus from $BV$  requires  ($spread(T)=2q$), ($size(T)=4q$) and  ($O(qn)$) moves.

\noindent{\bf Hypercubes(H).} According to the protocol, decontaminating a hypercube from $BV$s requires ($spread(H)=3$), ($size(H)=6$) and at maximum of $34$ moves.
 

 









