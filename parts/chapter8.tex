%%%%%%%%%%%%%%%%%%%%%%%%%%%
\chapter {Concluding Remarks and Future Work}
\label{INTRO}
%%%%%%%%%%%%%%%%%%%%%%%%%%%

\section{Summary}
Mobile agents are widely used in distributed and network systems, but  their employ can cause several security issues. Particularly, a malicious agent can cause computer nodes to malfunction or crash by contaminating or infecting them ({\em harmful agent}); a contaminated or infected host can  in turn destroy working agents for various malicious purposes ({\em harmful host}). A problem triggered by the harmful agent is called $Intruder\,Capture$(IC) which focuses on deploying a group of mobile agents to capture an extraneous mobile agent (the intruder) that moves arbitrarily fast through the network and infects the visiting sites. A problem triggered by the presence of  harmful hosts (also called $Black\,Holes$) is called $Black\, Hole\,Search$($BHS$), the focus of which is to locate the positions of these  $Black\,Holes$ in the topology. These two problems have been widely studied in many variants.

The BVD problem, integrating in its definition both the harmful aspects of the classical $BHS$ problem with the mobility of the classical $IC$ problem, was first introduced by \cite{cai}. The main focus of \cite{alotaibi,cai,cai1} \color{blue} add other references on BV \color{black} was   on the  design of sequential protocols for agents to complete this task   causing minimum network damage and   sacrifying the  minimum number of agents.  

In this thesis, we propose parallel strategies to deal with the BVD problem, focusing on employing more agents to explore the graph in parallel, thus decreasing the time needed to finish the decontamination.

This chapter highlights the main contributions of this thesis to the parallel BVD problem:

The first part of the thesis described the terminology and the model. In this thesis, we only consider the situation of $Fertile$ BV. In this case, we need to protect the network from further spread after detecting the original BV. A basic principle that monotonicity is necessary for optimality should be obeyed: in every network, to minimize the spread, a solution protocol must be monotone. 
%Also, we follow the same basic general strategy as \cite{cai}: a careful exploration of the network followed by an elimination.

The second part of this thesis focused on the parallel BVD problem for three important classes of interconnection network topologies:(multi-dimensional) grids, tori and chordal rings. For each topology, we provided a parallel strategy and analyzed the number of agents, the time cost, the number of movements and the casualties. Comparisons between the parallel strategy and the sequential strategy in each topology are also made showing that the better performances in terms of time are accompanying by a small increase in the number of agents employed.

The third part of this thesis focused on the parallel BVD problem in an arbitrary graph. We proposed two strategies to deal with the problem: $Flood\,Strategy$ and {\sc Castle-First} Strategy. The first strategy explores of the graphin optimal time (proportional to its diameter), but it employs  many more agents than the sequential strategy; on the other hand,  the {\sc Castle-First} Strategy  completes the exploration of the graph improving significantly the time complexity with respect to the sequential strategy,  but using a much lower number of agents   than the ones used in the $Flood\,Strategy$. In order to achieve this goal,  we had to  do some preprocessing  work: namely,  computing the length of the shortest route from the homebase to each node; using whiteboards to record some useful  information.

The last part of this thesis shifts the theoretical study on an arbitrary graph (the {\sc Castle-First} Strategy) to experimental investigation. A large number of simulations on different sizes of graphs with many connectivity densities are carried out. The experiment results show that even using one group of agents to explore the graph, 
the time cost by the  {\sc Castle-First} Strategy is much lower than the sequential strategy.
 
 Additionally, as we employ more agents than in the sequential strategy,
 when the connectivity of the graph reaches 30\% to 50\% 
 the {\sc Castle-First} Strategy using a single group of agents works most efficiently. 
 That is, the ratio between the time incurred by {\sc Castle-First} using a single group of agents and the time incurred by the sequential ({\sc Greedy})one  is much larger than the ratio between the number of agents used in {\sc Castle-First} with a single group and the number of agents used in {\sc Greedy}. 
 When the connectivity is low(less than 10\%), as the size of the graph grows, the advantage of using more groups of agents is becoming more evident. That is, when we use more groups of agents to explore the graph, the time cost still experiences a great decrease.


\section{Open Problems and Future Research}
The results of this thesis open many research problems and pose new questions.
\begin{enumerate}
\item In addition to grids, tori, and choral rings, future research can be extended to the design of parallel strategies for BVD  in other   network topologies, like the hypercube for example.
\item In the parallel strategies that we propose to deal with the BVD problem in the arbitrary graph, we have to do some preprocessing work (shortest path computation)  and we require full knowledge of the topology,  future research can be extended to deal the problem when less knowledge of the graph is available.
\item We   investigate the problem with   a single  BV,   future research can be extended to study the problem with more than one BV.
\end{enumerate}


















