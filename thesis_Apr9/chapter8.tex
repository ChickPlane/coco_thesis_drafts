%%%%%%%%%%%%%%%%%%%%%%%%%%%
\chapter {Conclusion}
\label{CON}
%%%%%%%%%%%%%%%%%%%%%%%%%%%

In  this thesis, we investigated the black virus disinfection problem in undirected chordal rings, presenting  a solution that is based on the use of the mobile agents model.
We addressed  the problem with the existence of  a single black virus at unknown location of the chordal ring that, when triggered, generates and sends new viruses to unprotected neighbours.  Although we assumed synchronous execution by the agents,  this strategy can be easily extended to work in asynchronous settings. 

The proposed solution is a distributed algorithm that consists of two phases: {\em Exploring and Shadowing} and {\em Surrounding and Eliminating}. Regarding cost, the efficiency measures considered include: the total number of black viruses originated in the system, the total number of mobile agents employed for disinfection and the total number of moves required by the agents. We found that the number of black viruses originated and agents required for disinfection is influenced by the location of the original black virus and that these numbers remain constant, regardless of the deployment method used in the surrounding phase. The number of moves varies according to the deployment strategy.


In our study we demonstrated that the first phase remains the same for all chordal rings and consists of the exploration of the graph using the safe exploration technique until the black virus is found. In order to achieve monotonicity, shadow agents are deployed during the search in order to guard the nodes that have been explored. The only difference between the types of chordal rings in the first phase is the number of shadow agents created at the beginning of the protocol. On the other hand, the second phase varies depending on the chord structure. Since black viruses are only destroyed when they arrive at a node that is being guarded, the agents must occupy all of the neighbouring nodes of the black viruses in the system. Routing is thus a critical part of this phase.
Assuming that the leading agent has full topological knowledge, the routing can be done globally by the leading agent. The leading agent always finds the shortest path to all targets and sends the surrounding agents through them. This routing method is optimal, yet it requires that the surrounding agents have larger memories. We also proposed other local methodologies that do not require that the entire path be calculated by the leading agent. We analyse the complexity af all the deployment strategies, providing upper bounds on the path lengths and optimality when possible.

In chapter  4, we addressed the black virus disinfection problem in double loop chordal rings. In order to disinfect the entire topology we require a maximum of 12 agents. The maximum number of black viruses is four. For the surrounding phase we presented three strategies:  move-optimal, simple greedy and smart greedy.  The move-optimal strategy is a global optimal deployment method that is mainly done by the leading agent. In this method, we demonstrated the optimal path to each target in all possible double loop cases.
For the simmple greedy and smart greedy strategies, we described two local approaches that allow the surrounding agents to decide their next step according to local information. These two strategies are not optimal.
%but require less agents capabilities: less storage.

In chapter 5, we described our solution in triple loop chordal rings. In order to disinfect the entire topology we require a maximum of 24 agents. The maximum number og black viruses originated is six. For the deployment phase we only described the move-optimal strategy since the greedy approach does not work for triple loops. We were unable to calculate the optimal paths for triple loops, however, we provided the upper bounds for the optimal loops. In order to getter a better understanding of the problem, we strudied two extreme triple loop cases and were able to find the optimal complexity for both cases. For simplicity, we only considered the shortly-chorded triple loop for the analysis of the surrounding phase. 

In chapter 6, we addressed the black virus dissinfection problem in  consecutive-chord loops. In order to disinfect the entire topology we require a maximum of $4k+2$ agents. The maximum number of black viruses originated is two,, where $k$ represents the longest chord. In the second phase we described a local strategy, the one-direction greedy strategy, which gives use the shortest paths to the targets.

In chapter 7, we discussed the black virus disinfection problem in general  chordal rings. In order to disinfect a chordal ring $C(d_1=1,\ldots, d_m)$ we require  $O(m^2)$ agents. The maximum number of black viruses orginated is  $2m$, where $m$ represents the number of chords in one direction. For the deployment phase we described a general protocol that is not optimal but that works correctly for any chord structure.

The following table summarizes the worst case complexities for the various chord structures considered in this thesis. The indicated complexity of the second phase corresponds with the best technique in terms of the number of moves required. 

\begin{center}
\begin{tabular}{|c|c|c|c|c|}\hline
        &{\bf $C(1,  k)$} &{\bf $C(1, p, k)$} & {\bf $C(1,2,\ldots,k)$}& {\bf $C(d_1,d_2,\ldots,d_m)$}
\\ \hline 
Agents & 12  & 24 & $4k+2$ &   $O(m^2)$  \\\hline
Spread & 4   & 6 &  $2k$ & $2m$  \\\hline
Moves Phase1 &$O(n)$ & $O(n)$ &  $O(nk) $ &  $O(nm) $  \\\hline
Moves Phase 2 & $\leq 36$   &     $\leq 78$  &  $O(k)$& $O(m^2)$ 
 \\\hline
\end{tabular}
\captionof{table}{A summary of the worst case complexities for various chordal ring types.}
\end{center}




As  previously mentioned, the problem of disinfecting a topology from black viruses is fairly new, and therefore, there remain many problems to be solved:
\begin{itemize}
\item In the case of   directed chordal rings, how can  the agents safely explore   in an asynchronous environment? In addition,  can  the routing always be performed correctly avoiding the black viruses ?
%\itemIn the case of asynchronous execution, we expla how it affects the overall complexity, what would be the effect of adding a third phase for termination?
\item In the case of  sequential activation, where the triggering of black viruses is done sequentially, what would be the effect on the number of agents and the number of moves?

\item In our study we only considered situations in which there is initially a single black virus. What would the topological impact be if we started with an unknown number of black viruses? In consecutive-chord loops, having two black viruses disconnects the graph. What about other typed of chordal rings? What would the minimum number of black viruses be? What would their locations be? Would they cause network disconnection?

\end{itemize}













